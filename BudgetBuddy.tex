\documentclass[
	% -- opções da classe memoir --
	12pt,				% tamanho da fonte
	oneside,			% para impressão em recto e verso. Oposto a oneside
	a4paper,			% tamanho do papel. 
	% -- opções da classe abntex2 --
	%chapter=TITLE,		% títulos de capítulos convertidos em letras maiúsculas
	%section=TITLE,		% títulos de seções convertidos em letras maiúsculas
	%subsection=TITLE,	% títulos de subseções convertidos em letras maiúsculas
	%subsubsection=TITLE,% títulos de subsubseções convertidos em letras maiúsculas
	% -- opções do pacote babel --
	english,			% idioma adicional para hifenização
	brazil,				% o último idioma é o principal do documento
	]{abntex2}
\usepackage[utf8]{inputenc}
\usepackage{longtable}
\usepackage{booktabs}

% Title Page
\title{BudgetBuddy - Organizador de Orçamento Pessoal}
\author{João Lucas Lucchetta}


\begin{document}
\maketitle

\begin{abstract}
\end{abstract}

\chapter*{Introdução}
A gerencia de gastos é um quesito importante da moderna. Serviços e bens diversos podem ser adquiridos com poucos toques numa tela,
e essa facilidade pode ameaçar o planejamento financeiro de muitos. Estudantes do ensino superior, geralmente, levam uma vida agitada
e não raramente negligenciam certos aspectos da vida adulta, como por exemplo seus orçamentos pessoais. \\ 
Este relatório detalha o funcionamento e especificações do \textit{BudgetBuddy}, um software de assistência a organização de orçamentos 
pessoais, desenvolvido baseando-se nas necessidades gerais da população estudantil do Ensino Superior Brasileiro.


\chapter{Problema}
[A SER ADICIONADO]
\chapter{Requisitos}
De modo a solucionar os problemas mencionados, a versão inicial do software necessita endereçar alguns requerimentos, chamados de Requisitos Funcionais e Não-Funcionais. Estes delimitam funções que a aplicação deve oferecer ou requisitos operacionais da aplicação.
\section{Requsitos Funcionais}
Esses são as funções consideradas essenciais ao software e que serão implementadas de acordo com seu nível de prioridades, tendo
primazia no período de desenvolvimento as funções com prioridade alta. Os requisitos não funcionais para essa aplicação, segundo análise realizada são expostos na Tabela \ref{table:ReqFunc}.
{
\renewcommand*{\arraystretch}{1.6}
\begin{longtable}{|c|p{7cm}|c|}
\hline
 \textbf{Nome da Atividade} & \textbf{Descrição} & \textbf{Prioridade} \\ \midrule
Adicionar Entrada          & A aplicação deve permitir a adição de valores ao orçamento                       & ALTA \\ 
Adicionar Saída            & A aplicação deve permitir a adição de valores negativos (despesas) ao orçamento  & ALTA \\
Adicionar Despesa Recorrente & A aplicação deve permitir a adição de despesas recorrentes (e.g. Contas de Água, Internet) & ALTA \\
Cadastrar Categorias       & A aplicação deve permitir o cadastro de categorias de movimentações (e.g. Salário, Faculdade) & ALTA \\
Categorizar Entradas       & A aplicação deve permitir a adição de uma categoria à entrada (e.g. Freelance, Bolsa-de-pesquisa) & ALTA \\
Categorizar Saídas       & A aplicação deve permitir a adição de uma categoria às saídas (e.g. Alimentação, Lazer) & ALTA \\
Editar Movimentações     & A aplicação deve permitir a edição das movimentações (Entradas, Saídas e Deepesas) & ALTA \\
Excluir Movimentações     & A aplicação deve permitir a removção das movimentações já adicionadas & ALTA \\
Visualizar Orçamento  & A aplicação deve permitir a visualização da estado atual do orçamento (mensal) & ALTA \\
Relatório em PDF      & A aplicação deve possibilitar a exportação de um relatório do orçamento atual em PDF & MÉDIA \\
Gráficos e Estatísticas  & A aplicação deve possibilitar a exportação de um relatório do orçamento atual em PDF & MÉDIA \\
Histórico de Movimentações   & A aplicação deve possibilitar a visualização de histórico entradas e saídas dos meses anteriores & BAIXA \\
\caption{Requisitos funcionais da aplicação}
\label{table:ReqFunc}
\end{longtable}
}


\bibliography{BudgetBuddyRefs}

\end{document}          

