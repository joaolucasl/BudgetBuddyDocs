\documentclass[
	% -- op��es da classe memoir --
	12pt,				% tamanho da fonte
	openright,			% cap�tulos come�am em p�g �mpar (insere p�gina vazia caso preciso)
	twoside,			% para impress�o em recto e verso. Oposto a oneside
	a4paper,			% tamanho do papel. 
	% -- op��es da classe abntex2 --
	%chapter=TITLE,		% t�tulos de cap�tulos convertidos em letras mai�sculas
	%section=TITLE,		% t�tulos de se��es convertidos em letras mai�sculas
	%subsection=TITLE,	% t�tulos de subse��es convertidos em letras mai�sculas
	%subsubsection=TITLE,% t�tulos de subsubse��es convertidos em letras mai�sculas
	% -- op��es do pacote babel --
	english,			% idioma adicional para hifeniza��o
	brazil,				% o �ltimo idioma � o principal do documento
	]{abntex2}
\usepackage[utf8]{inputenc}

% Title Page
\title{BudgetBuddy - Organizador de Or�amento Pessoal}
\author{Jo�o Lucas Lucchetta}


\begin{document}
\maketitle

\begin{abstract}
\end{abstract}

\section*{Introdu��o}
A gerencia de gastos � um quesito importante da moderna. Servi�os e bens diversos podem ser adquiridos com poucos toques numa tela,
e essa facilidade pode amea�ar o planejamento financeiro de muitos. Estudantes do ensino superior, geralmente, levam uma vida agitada
e n�o raramente negligenciam certos aspectos da vida adulta, como por exemplo seus or�amentos pessoais. 

\\ Este relat�rio detalha o funcionamento e especifica��es do \textit{BudgetBuddy}, um software de assist�ncia a organiza��o de or�amentos 
pessoais, desenvolvido baseando-se nas necessidades gerais da popula��o estudantil do Ensino Superior Brasileiro.

\section{Funcionalidades}


\bibliography{BudgetBuddyRefs}

\end{document}          
